\newpage
\begin{center}
  \textbf{\large АННОТАЦИЯ}
\end{center}

\onehalfspacing
\setcounter{page}{2}

Данная работа посвящена разработке и исследованию метода управления на основе Model Predictive Path Integral (MPPI), обеспечивающего точное выполнение линейных по ускорениям ограничений (Пфаффовых ограничений) вида $A(q, v, t)\,\dot{v} = b(q, v, t)$, где $q \in \mathbb{R}^n$ --- обобщённые координаты, $v$ --- обобщённые скорости, $A \in \mathbb{R}^{m \times n}$ --- матрица ограничений, $b \in \mathbb{R}^m$ --- вектор правой части.

Предлагаемый подход (Gauss-MPPI) основан на интеграции принципа наименьшего принуждения Гаусса в процедуру семплирования траекторий MPPI, что позволяет гарантировать выполнение ограничений на каждом шаге интегрирования динамики системы. Для проекции свободного ускорения на множество допустимых ускорений используется аналитическое решение --- уравнение Удвадия--Калабы.

Работа включает обзор существующих методов обработки ограничений в MPPI, теоретическое обоснование предложенного подхода, программную реализацию алгоритма с поддержкой параллельных вычислений, а также экспериментальную проверку на задачах робототехники: неголономный мобильный робот, робот-манипулятор с ограничениями в операционном пространстве и система с контактными взаимодействиями.

% TODO: обновить аннотацию по результатам завершённых экспериментов

\newpage
\renewcommand{\contentsname}{\centerline{\large СОДЕРЖАНИЕ}}
\tableofcontents

\newpage
\begin{center}
  \textbf{\large ВВЕДЕНИЕ}
\end{center}
\addcontentsline{toc}{chapter}{ВВЕДЕНИЕ}

Задача управления механическими системами в условиях ограничений является одной из центральных в современной робототехнике. Многие практически важные системы --- мобильные роботы, манипуляторы, шагающие роботы --- подчиняются кинематическим и динамическим ограничениям, которые должны выполняться точно на каждом шаге управления.

Model Predictive Path Integral (MPPI) --- метод модельно-предиктивного управления, основанный на стохастической выборке траекторий. MPPI не требует вычисления градиентов функции стоимости, работает с невыпуклыми и негладкими целевыми функциями и эффективно параллелизуется на GPU. Однако исходная формулировка MPPI не предусматривает явной обработки ограничений.

Существующие подходы к обработке ограничений в MPPI --- штрафные методы, барьерные функции, вероятностные фильтры --- ориентированы преимущественно на ограничения типа неравенств (безопасность, границы управления). Ограничения равенств на уровне ускорений (Пфаффовы ограничения) в контексте MPPI практически не исследованы.

\textbf{Целью данной работы} является разработка и исследование метода управления Gauss-MPPI, обеспечивающего точное выполнение линейных по ускорениям ограничений вида:
\begin{equation}
    A(q, v, t) \, \dot{v} = b(q, v, t).
\end{equation}

Предлагаемый подход основан на интеграции принципа наименьшего принуждения Гаусса в процедуру семплирования траекторий MPPI, что позволяет гарантировать выполнение ограничений на каждом шаге интегрирования динамики.

Для достижения поставленной цели необходимо решить следующие \textbf{задачи}:
\begin{enumerate}
    \item Провести анализ существующих методов обработки ограничений в MPPI, выявить их преимущества и недостатки.
    \item Разработать теоретические основы метода Gauss-MPPI: сформулировать задачу проекции свободного ускорения на множество допустимых ускорений как задачу квадратичного программирования.
    \item Исследовать применимость принципа наименьшего принуждения Гаусса для стохастических дифференциальных уравнений.
    \item Разработать алгоритм Gauss-MPPI и его программную реализацию с поддержкой параллельных вычислений на GPU.
    \item Провести экспериментальную проверку на модельных задачах робототехники.
    \item Выполнить сравнительный анализ с классическим MPPI, CBF-MPPI и штрафными методами.
\end{enumerate}

\newpage
