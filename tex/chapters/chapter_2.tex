\newpage
\begin{center}
  \textbf{\large 2. МЕТОДОЛОГИЯ}
\end{center}
\refstepcounter{chapter}
\addcontentsline{toc}{chapter}{2. МЕТОДОЛОГИЯ}

\section{Каноническое уравнение манипулятора}

Динамика робота-манипулятора с $n$ степенями свободы описывается каноническим уравнением:
\begin{equation}
    M(q)\dot{v} + c(q, v) = B\tau,
    \label{eq:manipulator}
\end{equation}
где $q \in \mathbb{R}^n$ --- вектор обобщённых координат, $v = \dot{q} \in \mathbb{R}^n$ --- вектор обобщённых скоростей, $M(q) \in \mathbb{R}^{n \times n}$ --- матрица масс (симметричная, положительно определённая), $c(q, v) \in \mathbb{R}^n$ --- вектор кориолисовых, центробежных и гравитационных сил, $B \in \mathbb{R}^{n \times m}$ --- матрица распределения управляющих воздействий, $\tau \in \mathbb{R}^m$ --- вектор управляющих моментов.

Для полностью актуированной системы ($m = n$, $B = I$):
\begin{equation}
    M(q)\dot{v} = \tau - c(q, v).
    \label{eq:manipulator_full}
\end{equation}

Свободное ускорение --- ускорение системы при заданном управлении без учёта ограничений:
\begin{equation}
    \dot{v}_{\text{free}} = M(q)^{-1}(\tau - c(q, v)).
    \label{eq:free_accel}
\end{equation}

Состояние системы определяется парой $x = (q, v) \in \mathbb{R}^{2n}$. Эволюция состояния при заданном ускорении $\dot{v}$ описывается полуявной (симплектической) схемой Эйлера:
\begin{equation}
\begin{aligned}
    v_{k+1} &= v_k + \dot{v}_k \cdot \Delta t, \\
    q_{k+1} &= q_k + v_{k+1} \cdot \Delta t,
\end{aligned}
    \label{eq:semi_implicit}
\end{equation}
которая обеспечивает лучшее сохранение энергии по сравнению с явной схемой.

\section{Динамическая модель в алгоритме MPPI}

В алгоритме MPPI динамическая модель \eqref{eq:manipulator} используется для прямого моделирования (rollout) траекторий системы. На каждой итерации генерируется $K$ возмущённых последовательностей управления:
\begin{equation}
    \tau_t^{(k)} = \tau_t^{\text{nom}} + \epsilon_t^{(k)}, \quad \epsilon_t^{(k)} \sim \mathcal{N}(0, \Sigma),
    \label{eq:perturbed_control}
\end{equation}
где $\tau_t^{\text{nom}}$ --- номинальное управление на шаге $t$, $\Sigma \in \mathbb{R}^{m \times m}$ --- ковариационная матрица шума.

Для каждой возмущённой последовательности выполняется прямое моделирование на горизонте $T$ шагов: вычисляется свободное ускорение $\dot{v}_{\text{free}} = M(q)^{-1}(\tau_t^{(k)} - c(q, v))$, интегрируется состояние по схеме \eqref{eq:semi_implicit} и накапливается стоимость. Полная стоимость $k$-й траектории:
\begin{equation}
    S_k = \phi(x_T^{(k)}) + \sum_{t=0}^{T-1} \mathcal{L}(x_t^{(k)}, \tau_t^{(k)}),
    \label{eq:trajectory_cost}
\end{equation}
где $\phi$ --- терминальная стоимость, $\mathcal{L}$ --- мгновенная стоимость.

Номинальное управление обновляется взвешенным усреднением возмущений:
\begin{equation}
    \tau_t^{\text{nom}} \gets \tau_t^{\text{nom}} + \sum_{k=1}^{K} w_k \, \epsilon_t^{(k)}, \quad
    w_k = \frac{\exp(-(S_k - \beta) / \lambda)}{\sum_{j=1}^{K} \exp(-(S_j - \beta) / \lambda)},
    \label{eq:mppi_update}
\end{equation}
где $\lambda > 0$ --- параметр температуры, $\beta = \min_k S_k$ --- нормировка для численной устойчивости.

\section{Интеграция Пфаффовых ограничений через принцип Гаусса}

Рассмотрим систему \eqref{eq:manipulator}, подчинённую Пфаффовым ограничениям, линейным по ускорениям:
\begin{equation}
    A(q, v, t)\dot{v} = b(q, v, t),
    \label{eq:pfaffian}
\end{equation}
где $A \in \mathbb{R}^{p \times n}$ --- матрица ограничений, $b \in \mathbb{R}^p$ --- вектор правой части, $p$ --- число ограничений. Такие ограничения естественно возникают в робототехнике:
\begin{itemize}
    \item Неголономные системы: ограничение бокового проскальзывания колёс $\dot{x}\sin\theta - \dot{y}\cos\theta = 0$;
    \item Ограничения в операционном пространстве: $J(q)\dot{v} = \ddot{x}_d - \dot{J}v$;
    \item Контактные ограничения: нормальная компонента ускорения в точке контакта равна нулю.
\end{itemize}

Ограничения \eqref{eq:pfaffian} могут быть \textit{виртуальными} --- не воплощёнными в физике среды, а задаваемыми проектировщиком (например, движение конечного эффектора строго по прямой). В таком случае среда не обеспечивает их выполнение, и контроллер должен выдавать управляющий сигнал, который сам по себе порождает допустимое движение.

Согласно принципу наименьшего принуждения Гаусса, ускорение системы при наличии ограничений \eqref{eq:pfaffian} определяется как решение задачи квадратичного программирования:
\begin{equation}
\begin{aligned}
    &\min_{\dot{v}} && \frac{1}{2}[\dot{v} - \dot{v}_{\text{free}}]^T M(q) [\dot{v} - \dot{v}_{\text{free}}] \\
    &\text{s.t.} && A(q, v, t)\dot{v} = b(q, v, t),
\end{aligned}
    \label{eq:gauss_qp}
\end{equation}
где $\dot{v}_{\text{free}}$ --- свободное ускорение \eqref{eq:free_accel}. Физический смысл: из всех ускорений, удовлетворяющих ограничениям, выбирается ближайшее к свободному в метрике, определяемой матрицей масс.

Аналитическое решение задачи \eqref{eq:gauss_qp} даётся уравнением Удвадия--Калабы:
\begin{equation}
    \dot{v} = \dot{v}_{\text{free}} + M^{-1/2}(AM^{-1/2})^{+}(b - A\dot{v}_{\text{free}}),
    \label{eq:udwadia_kalaba}
\end{equation}
где $(\cdot)^+$ --- псевдообратная матрица Мура--Пенроуза. Решение допускает декомпозицию:
\begin{equation}
    \dot{v} = \dot{v}_{\text{free}} + M^{-1}\lambda_c, \quad \lambda_c = M^{1/2}(AM^{-1/2})^{+}(b - A\dot{v}_{\text{free}}),
    \label{eq:constraint_force}
\end{equation}
где $\lambda_c$ --- обобщённая сила связи, минимальная в смысле принуждения Гаусса.

Проекция Гаусса используется в алгоритме в двух ролях:
\begin{enumerate}
    \item \textbf{Во время rollout}: проекция ускорений обеспечивает моделирование физически корректных (удовлетворяющих ограничениям) траекторий, что необходимо для правильной оценки стоимости и вычисления весов.
    \item \textbf{На выходе алгоритма}: проекция номинального управления обеспечивает формирование выходного сигнала, содержащего силу связи. Без этого шага возвращаемое управление не будет связано с наложенными ограничениями.
\end{enumerate}

\section{Алгоритм Gauss-MPPI}

Полный алгоритм Gauss-MPPI представлен в виде псевдокода (Алгоритм~\ref{alg:gauss_mppi}).

\begin{algorithm}[H]
\caption{Gauss-MPPI}
\label{alg:gauss_mppi}
\begin{algorithmic}[1]
\Require $x_0 = (q_0, v_0)$, $U = \{\tau_0, \ldots, \tau_{T-1}\}$, $K$, $\lambda$, $\Sigma$, $\Delta t$
\Ensure $\tau_{\text{out}}$
\Statex \textit{--- Rollout с проекцией Гаусса ---}
\For{$k = 1, \ldots, K$} \textbf{параллельно}
    \State $(q, v) \gets (q_0, v_0)$, \quad $S_k \gets 0$
    \For{$t = 0, \ldots, T-1$}
        \State $\epsilon_t^{(k)} \sim \mathcal{N}(0, \Sigma)$
        \State $\hat{\tau} \gets \tau_t + \epsilon_t^{(k)}$
        \State $\dot{v}_{\text{free}} \gets M(q)^{-1}(\hat{\tau} - c(q, v))$
        \State $\dot{v} \gets \dot{v}_{\text{free}} + M^{-1/2}(AM^{-1/2})^{+}(b - A\dot{v}_{\text{free}})$ \Comment{проекция}
        \State $v \gets v + \dot{v} \cdot \Delta t$, \quad $q \gets q + v \cdot \Delta t$
        \State $S_k \gets S_k + \mathcal{L}((q, v),\, \hat{\tau})$
    \EndFor
    \State $S_k \gets S_k + \phi(q, v)$
\EndFor
\Statex \textit{--- Обновление номинального управления ---}
\State $\beta \gets \min_k S_k$
\For{$k = 1, \ldots, K$}
    \State $w_k \gets \exp\bigl(-(S_k - \beta) / \lambda\bigr)$
\EndFor
\State $\eta \gets \sum_{k=1}^{K} w_k$
\For{$t = 0, \ldots, T-1$}
    \State $\tau_t \gets \tau_t + \frac{1}{\eta}\sum_{k=1}^{K} w_k \, \epsilon_t^{(k)}$
\EndFor
\Statex \textit{--- Проекция выходного управления ---}
\State $\dot{v}_{\text{free}} \gets M(q_0)^{-1}(\tau_0 - c(q_0, v_0))$
\State $\lambda_c \gets M(q_0)^{1/2}\bigl(A_0 M(q_0)^{-1/2}\bigr)^{+}\bigl(b_0 - A_0 \dot{v}_{\text{free}}\bigr)$
\State $\tau_{\text{out}} \gets \tau_0 + \lambda_c$
\State \Return $\tau_{\text{out}}$
\end{algorithmic}
\end{algorithm}

Алгоритм состоит из трёх этапов. Первый этап (строки 1--12) --- rollout с проекцией Гаусса: каждая семплированная траектория моделируется с учётом ограничений, что обеспечивает корректную оценку стоимости. Второй этап (строки 13--19) --- стандартное обновление MPPI: номинальная последовательность $U$ обновляется взвешенным усреднением возмущений. Третий этап (строки 20--23) --- проекция выходного управления: к номинальному моменту $\tau_0$ добавляется обобщённая сила связи $\lambda_c$, вычисленная в текущем состоянии $(q_0, v_0)$.

Номинальная последовательность $U$ хранит \textit{свободные} (без силы связи) управления. Это принципиально: сила связи $\lambda_c$ зависит от состояния системы, которое на горизонте планирования неизвестно. При rollout проекция вычисляется в каждой точке траектории, а при выдаче управления --- в текущем известном состоянии.

\section{Обеспечение ограничений в Gauss-MPPI}

\subsection{Декомпозиция выходного управления}

Выходной сигнал $\tau_{\text{out}}$ допускает декомпозицию:
\begin{equation}
    \tau_{\text{out}} = \underbrace{\tau_0^{\text{nom}}}_{\text{MPPI-оптимальное}} + \underbrace{\lambda_c(q_0, v_0, \tau_0^{\text{nom}})}_{\text{сила связи}},
    \label{eq:tau_decomposition}
\end{equation}
где $\tau_0^{\text{nom}}$ --- номинальное управление, найденное MPPI-оптимизацией, а $\lambda_c$ --- минимальная (в смысле принуждения Гаусса) сила, необходимая для выполнения ограничений.

При подаче $\tau_{\text{out}}$ на неограниченную систему $M\dot{v} + c = \tau_{\text{out}}$ результирующее ускорение:
\begin{equation}
    \dot{v} = M^{-1}(\tau_{\text{out}} - c) = M^{-1}(\tau_0^{\text{nom}} + \lambda_c - c) = \dot{v}_{\text{free}} + M^{-1}\lambda_c,
    \label{eq:applied_accel}
\end{equation}
что в точности совпадает с ограниченным ускорением \eqref{eq:udwadia_kalaba}.

\subsection{Доказательство выполнения ограничений}

\textbf{Утверждение.} Ускорение \eqref{eq:applied_accel}, порождаемое выходным управлением $\tau_{\text{out}}$, удовлетворяет ограничению $A\dot{v} = b$.

\textit{Доказательство.} Подставим \eqref{eq:applied_accel} в $A\dot{v}$:
\begin{equation}
    A\dot{v} = A\dot{v}_{\text{free}} + AM^{-1/2}(AM^{-1/2})^{+}(b - A\dot{v}_{\text{free}}).
\end{equation}
Обозначим $\Phi = AM^{-1/2}$. Тогда:
\begin{equation}
    A\dot{v} = A\dot{v}_{\text{free}} + \Phi\Phi^{+}(b - A\dot{v}_{\text{free}}).
\end{equation}
Если $A$ имеет полный строчный ранг, то $\Phi\Phi^+ = I_p$ и:
\begin{equation}
    A\dot{v} = A\dot{v}_{\text{free}} + b - A\dot{v}_{\text{free}} = b.
\end{equation}

Равенство выполняется \textit{для любого} $\tau_0^{\text{nom}} \in \mathbb{R}^m$. Следовательно, независимо от результата взвешенного усреднения MPPI, выходной сигнал $\tau_{\text{out}}$ порождает допустимое ускорение.

\subsection{Роль проекции во время rollout}

Проекция Гаусса во время rollout не влияет на выходное управление непосредственно, но обеспечивает \textit{корректную оценку стоимости}. Без проекции:
\begin{itemize}
    \item Семплированные траектории нарушали бы ограничения;
    \item Стоимости $S_k$ не отражали бы реальное поведение системы под ограничениями;
    \item Веса $w_k$ присваивались бы на основе нефизичных траекторий;
    \item MPPI-оптимизация находила бы управление, не учитывающее влияние ограничений на динамику.
\end{itemize}

Таким образом, проекция при rollout обеспечивает качество (оптимальность) управления, а проекция на выходе --- выполнение ограничений.

\section{Оптимальность Gauss-MPPI}

Покажем, что Gauss-MPPI сходится к оптимальному управлению для задачи с ограничениями.

\subsection{Ограниченная динамика как модель MPPI}

Задача ограниченного оптимального управления --- минимизация функционала \eqref{eq:trajectory_cost} при динамике \eqref{eq:manipulator_full} и ограничениях \eqref{eq:pfaffian}:
\begin{equation}
\begin{aligned}
    &\min_{U} && J(U) = \phi(x_T) + \sum_{t=0}^{T-1} \mathcal{L}(x_t, \tau_t) \\
    &\text{s.t.} && M\dot{v}_t + c = \tau_t, \quad A\dot{v}_t = b, \quad t = 0, \ldots, T-1.
\end{aligned}
    \label{eq:constrained_ocp}
\end{equation}

Подставив уравнение Удвадия--Калабы \eqref{eq:udwadia_kalaba}, ограничения на ускорения элиминируются, и задача \eqref{eq:constrained_ocp} эквивалентна задаче без явных ограничений с модифицированной динамикой:
\begin{equation}
    \min_{U} J(U), \quad \text{s.t.} \quad \dot{v}_t = f_c(x_t, \tau_t),
    \label{eq:reduced_ocp}
\end{equation}
где $f_c$ --- ограниченная динамика, задаваемая формулой \eqref{eq:udwadia_kalaba}. Задача \eqref{eq:reduced_ocp} не содержит явных ограничений: выполнение $A\dot{v} = b$ гарантируется структурой $f_c$.

Gauss-MPPI при rollout использует именно $f_c$ в качестве модели динамики. Следовательно, с точки зрения MPPI, задача \eqref{eq:reduced_ocp} есть стандартная задача оптимального управления с динамикой $f_c$.

\subsection{Аффинность ограниченной динамики по управлению}

\textbf{Утверждение.} Ограниченная динамика $f_c$ аффинна по $\tau$.

\textit{Доказательство.} Подставим $\dot{v}_{\text{free}} = M^{-1}(\tau - c)$ в \eqref{eq:udwadia_kalaba}, обозначив $\Phi = AM^{-1/2}$:
\begin{equation}
\begin{aligned}
    \dot{v} &= M^{-1}(\tau - c) + M^{-1/2}\Phi^{+}\bigl(b - \Phi M^{-1/2}(\tau - c)\bigr) \\
    &= \bigl(I - M^{-1/2}\Phi^{+}\Phi M^{-1/2}\bigr)M^{-1}(\tau - c) + M^{-1/2}\Phi^{+}b.
\end{aligned}
\end{equation}
Обозначим:
\begin{equation}
    N(q, v, t) = \bigl(I - M^{-1/2}\Phi^{+}\Phi M^{-1/2}\bigr)M^{-1}, \quad h(q, v, t) = -Nc + M^{-1/2}\Phi^{+}b.
    \label{eq:Nh_def}
\end{equation}
Тогда:
\begin{equation}
    f_c(x, \tau) = N(x, t)\,\tau + h(x, t),
    \label{eq:fc_affine}
\end{equation}
где $N$ и $h$ зависят от $(q, v, t)$, но не от $\tau$.

Матрица $N$ имеет ясную интерпретацию: $M^{-1/2}\Phi^+\Phi M^{-1/2}$ --- проектор (в метрике $M$) на подпространство, связанное ограничениями, а $(I - M^{-1/2}\Phi^+\Phi M^{-1/2})$ --- проектор на ортогональное дополнение. Таким образом, $N$ пропускает только ту компоненту управления, которая не противоречит ограничениям.

\subsection{Сходимость к оптимальному ограниченному управлению}

Теория path integral control~\cite{kappen2005path, theodorou2010generalized} устанавливает следующий результат. Для системы с аффинной по управлению динамикой $\dot{v} = g(x) + G(x)\tau$, стоимостью вида $J = \phi(x_T) + \sum_{t}[\ell(x_t) + \frac{1}{2}\tau_t^T R\,\tau_t]$ и условием согласованности шума $\Sigma = \lambda R^{-1}$, оптимальное управление выражается через интеграл по траекториям:
\begin{equation}
    \tau_t^* = \lim_{K \to \infty} \frac{\sum_{k=1}^{K} \tau_t^{(k)} \exp(-S_k / \lambda)}{\sum_{k=1}^{K} \exp(-S_k / \lambda)},
    \label{eq:pi_optimal}
\end{equation}
что является в точности оценкой MPPI.

Поскольку ограниченная динамика $f_c$ аффинна по $\tau$ (формула \eqref{eq:fc_affine}), а шум управления $\epsilon \sim \mathcal{N}(0, \Sigma)$ входит через тот же канал, что и управление ($f_c(x, \tau + \epsilon) = N(\tau + \epsilon) + h$), условие согласованности шума выполняется. Следовательно, при $K \to \infty$ оценка MPPI сходится к оптимальному управлению задачи \eqref{eq:reduced_ocp}.

Для произвольных (не квадратичных по $\tau$) функций стоимости MPPI минимизирует свободную энергию:
\begin{equation}
    F_\lambda(U) = -\lambda \log \mathbb{E}_{E \sim \mathcal{N}(0, \Sigma)}\left[\exp\!\left(-\frac{J(U + E)}{\lambda}\right)\right],
    \label{eq:free_energy}
\end{equation}
являющуюся верхней оценкой на $\min_E J(U + E)$ (по неравенству Йенсена). При $\lambda \to 0$ свободная энергия сходится к истинному минимуму: $F_\lambda(U) \to \min_E J(U + E)$.

\textbf{Итог.} Gauss-MPPI сходится к оптимальному управлению задачи \eqref{eq:constrained_ocp} при $K \to \infty$ (точность оценки) и $\lambda \to 0$ (точность аппроксимации). Проекция на выходе (строки 20--23 Алгоритма~\ref{alg:gauss_mppi}) не изменяет оптимальность: она детерминированно преобразует $\tau_0^{\text{nom}}$ в $\tau_{\text{out}} = \tau_0^{\text{nom}} + \lambda_c$, порождающий ту же ограниченную траекторию при подаче на неограниченную систему.

\section{Стабилизация ограничений}

При численном интегрировании ограничения на уровне ускорений \eqref{eq:pfaffian} гарантируют $A\dot{v} = b$ на каждом шаге, но не предотвращают накопление ошибок на уровне координат и скоростей. Если ограничение \eqref{eq:pfaffian} получено дифференцированием ограничений более высокого уровня, возникает дрейф.

\subsection{Дрейф голономных ограничений}

Рассмотрим голономное ограничение $\varphi(q) = 0$. Дифференцирование по времени даёт:
\begin{equation}
\begin{aligned}
    \dot{\varphi} &= \frac{\partial \varphi}{\partial q} v = 0, \\
    \ddot{\varphi} &= \frac{\partial \varphi}{\partial q} \dot{v} + \frac{d}{dt}\!\left(\frac{\partial \varphi}{\partial q}\right) v = 0.
\end{aligned}
    \label{eq:holonomic_diff}
\end{equation}
Ограничение на уровне ускорений $A\dot{v} = b$ с $A = \partial\varphi/\partial q$ и $b = -\dot{A}v$ эквивалентно $\ddot{\varphi} = 0$. Численное интегрирование гарантирует лишь приближённое выполнение $\ddot{\varphi} \approx 0$, при этом ошибки в $\varphi$ и $\dot{\varphi}$ могут накапливаться.

\subsection{Стабилизация по Баумгарте}

Метод Баумгарте~\cite{baumgarte1972} заменяет условие $\ddot{\varphi} = 0$ стабилизированным уравнением:
\begin{equation}
    \ddot{\varphi} + 2\alpha\dot{\varphi} + \beta^2\varphi = 0,
    \label{eq:baumgarte}
\end{equation}
где $\alpha, \beta > 0$ --- параметры стабилизации. Уравнение \eqref{eq:baumgarte} представляет собой уравнение демпфированного осциллятора для ошибки ограничения: при $\alpha = \beta$ (критическое демпфирование) отклонение $\varphi$ экспоненциально убывает.

Модифицированная правая часть ограничения на уровне ускорений:
\begin{equation}
    A\dot{v} = b - 2\alpha A v - \beta^2 \varphi(q).
    \label{eq:baumgarte_accel}
\end{equation}

Для ограничений на уровне скоростей $\Phi(q, v) = 0$ (неголономные ограничения) стабилизация принимает вид:
\begin{equation}
    A\dot{v} = b - \alpha\,\Phi(q, v),
    \label{eq:baumgarte_velocity}
\end{equation}
где $A$ и $b$ получены дифференцированием $\Phi$ по времени.

\subsection{Применение в Gauss-MPPI}

В алгоритме Gauss-MPPI стабилизация по Баумгарте интегрируется путём замены правой части $b$ на стабилизированную $b_{\text{stab}}$ --- как при rollout, так и при проекции выходного управления:
\begin{equation}
    \dot{v} = \dot{v}_{\text{free}} + M^{-1/2}(AM^{-1/2})^{+}(b_{\text{stab}} - A\dot{v}_{\text{free}}).
    \label{eq:gauss_stabilized}
\end{equation}

При использовании стабилизации выходное управление не обеспечивает точное выполнение исходного ограничения $\varphi(q) = 0$ мгновенно, но \textit{асимптотически стремится} к нему: динамика ошибки $\ddot{\varphi} + 2\alpha\dot{\varphi} + \beta^2\varphi = 0$ гарантирует экспоненциальное затухание отклонения с характерным временем $\sim 1/\alpha$.

Выбор параметров $\alpha$ и $\beta$ определяет скорость коррекции дрейфа. На практике используются значения порядка $\alpha = \beta \sim 1/\Delta t$, что обеспечивает коррекцию ошибки за один--два шага интегрирования.
