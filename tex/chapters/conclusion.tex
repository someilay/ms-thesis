\newpage
\begin{center}
  \textbf{\large ЗАКЛЮЧЕНИЕ}
\end{center}
\refstepcounter{chapter}
\addcontentsline{toc}{chapter}{ЗАКЛЮЧЕНИЕ}

% TODO: написать заключение по результатам работы

В данной работе рассмотрена задача управления механическими системами с линейными по ускорениям ограничениями методом MPPI. Предложен метод Gauss-MPPI, интегрирующий принцип наименьшего принуждения Гаусса в процедуру семплирования траекторий.

Основные результаты работы:
\begin{enumerate}
    \item Проведён анализ существующих методов обработки ограничений в MPPI, выявлен пробел в обработке ограничений равенств на уровне ускорений.
    \item Предложен метод Gauss-MPPI, использующий уравнение Удвадия--Калабы для проекции свободного ускорения на множество допустимых ускорений.
    \item Выполнена программная реализация классического MPPI в среде MuJoCo.
\end{enumerate}

% TODO: дополнить список результатов по завершении экспериментов
% TODO: добавить выводы о преимуществах и ограничениях метода
% TODO: указать направления дальнейших исследований
