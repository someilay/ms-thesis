\newpage
\begin{center}
    \textbf{\large 3. ПРОГРАММНАЯ РЕАЛИЗАЦИЯ И ЭКСПЕРИМЕНТЫ}
\end{center}
\refstepcounter{chapter}
\addcontentsline{toc}{chapter}{3. ПРОГРАММНАЯ РЕАЛИЗАЦИЯ И ЭКСПЕРИМЕНТЫ}

\section{Программный стек}

Реализация алгоритма Gauss-MPPI выполнена на языке Python с использованием следующих библиотек:
\begin{itemize}
    \item \textbf{JAX} --- фреймворк для дифференцируемых вычислений с JIT-компиляцией и автоматическим распараллеливанием (\texttt{jax.vmap}) на GPU. Используется для реализации MPPI: параллельная выборка $K$ траекторий, вычисление стоимостей и весов;
    \item \textbf{MuJoCo / MJX} --- физический движок MuJoCo и его JAX-совместимая обёртка MJX. MJX позволяет выполнять шаги симуляции на GPU, что критично для параллельного rollout в MPPI;
    \item \textbf{Pinocchio}~\cite{Pinocchio} --- библиотека для вычисления прямой кинематики, якобианов и их производных по времени. Используется для задач, требующих аналитического доступа к кинематическим величинам (эксперимент с Dual-arm YuMi).
\end{itemize}

\section{Реализация алгоритма Gauss-MPPI}

\subsection{Контроллер MPPI}

Реализован класс \texttt{MPPI}, принимающий на вход конфигурацию (число траекторий $K$, горизонт $T$, температура $\lambda$, ковариация шума $\Sigma$) и функцию rollout. На каждом вызове \texttt{command()}:
\begin{enumerate}
    \item Генерируются $K$ возмущений $\epsilon^{(k)} \sim \mathcal{N}(0, \Sigma)$;
    \item Возмущённые управления подаются на функцию rollout, возвращающую стоимости;
    \item Вычисляются веса $w_k$ и обновляется номинальная последовательность;
    \item Возвращается первый элемент обновлённой последовательности.
\end{enumerate}

Контроллер реализован полностью в JAX и JIT-компилируется для работы на GPU.

\subsection{Ограниченная динамика}

Проекция Гаусса реализована в классе \texttt{MjxWrapper}. После вычисления свободного ускорения $\dot{v}_{\text{free}}$ средствами MJX, выполняется проекция по уравнению Удвадия--Калабы:
\begin{enumerate}
    \item Из каждого ограничения вычисляются блоки $A_i$, $b_i$;
    \item Блоки объединяются: $A = [A_1; \ldots; A_r]$, $b = [b_1; \ldots; b_r]$;
    \item Вычисляются $M^{1/2}$, $M^{-1/2}$ через собственное разложение матрицы масс;
    \item Сила связи $\lambda_c = M^{1/2}(AM^{-1/2})^+(b - A\dot{v}_{\text{free}})$ добавляется к обобщённым силам.
\end{enumerate}

Каждое ограничение реализуется как подкласс абстрактного класса \texttt{Constraint}, предоставляющий метод \texttt{compute(q, v)} $\to$ $(A, b)$. Это позволяет комбинировать несколько ограничений в одном эксперименте.

\subsection{Проекция выходного управления}

Согласно Алгоритму~\ref{alg:gauss_mppi}, после обновления номинальной последовательности MPPI выходное управление проецируется в текущем состоянии:
\begin{equation}
    \tau_{\text{out}} = \tau_0^{\text{nom}} + \lambda_c(q_0, v_0, \tau_0^{\text{nom}}).
\end{equation}
Данная проекция вычисляется тем же механизмом, что и проекция при rollout, но однократно для текущего состояния $(q_0, v_0)$.

% TODO: реализовать проекцию выходного управления как отдельный вызов
% TODO: профилировать накладные расходы проекции относительно rollout

\section{Эксперимент 1: Колёсный робот без проскальзывания}

\subsection{Описание системы}

Рассматривается дифференциально-приводной мобильный робот Boxer --- платформа с двумя ведущими колёсами радиуса $r = 0.08$~м. Робот моделируется как свободное тело ($n_v = 6 + 2 = 8$: 6 DOF базы + 2 вращения колёс) с двумя группами виртуальных ограничений.

\textbf{1. Ограничение горизонтальной плоскости.} Робот удерживается на высоте $z = z_{\text{ref}}$ с фиксированной ориентацией вертикальной оси. Ограничение задаётся тремя скалярными условиями:
\begin{equation}
    c_z = z_{\text{anchor}} - z_{\text{ref}} = 0, \quad c_{\text{orient}} = (\text{rot}(\text{anchor}) \cdot \hat{r})_{xy} = 0,
\end{equation}
где $\hat{r}$ --- начальное направление вертикальной оси. Стабилизация по Баумгарте:
\begin{equation}
    A_1 \dot{v} = -\dot{A}_1 v - K_p^{(1)} c - K_d^{(1)} \dot{c}.
\end{equation}

\textbf{2. Ограничение отсутствия проскальзывания.} Для каждого колеса $i$ скорость точки контакта с поверхностью в проекции на плоскость равна нулю:
\begin{equation}
    v_{\text{lin},i}^{xy} - \dot{\theta}_i \cdot r \cdot (a_i \times e_z)^{xy} = 0,
    \label{eq:no_slip}
\end{equation}
где $v_{\text{lin},i}$ --- линейная скорость центра колеса в системе координат родительского тела, $\dot{\theta}_i$ --- угловая скорость вращения колеса, $a_i$ --- ось вращения. Условие \eqref{eq:no_slip} --- ограничение на уровне скоростей; при дифференцировании по времени получается ограничение вида $A_2\dot{v} = b_2$ со стабилизацией:
\begin{equation}
    A_2 \dot{v} = -\dot{A}_2 v - K_d^{(2)} c_w.
\end{equation}

Суммарно система имеет $p = 3 + 4 = 7$ ограничений на $n_v = 8$ степеней свободы, оставляя одну управляемую степень свободы для каждого колеса (вращение).

\subsection{Задача управления}

Робот должен достичь заданной целевой позиции $(x_d, y_d)$ на плоскости. Функция стоимости:
\begin{equation}
    \mathcal{L}(x, \tau) = w_{\text{pos}} \| p_{xy} - p_d \|^2 + w_{\tau} \| \tau \|^2,
\end{equation}
где $p_{xy}$ --- текущая позиция робота на плоскости, $p_d$ --- целевая позиция.

Управление $\tau \in \mathbb{R}^{n_v}$ --- обобщённые силы. Проекция Гаусса гарантирует, что результирующее ускорение удовлетворяет ограничениям отсутствия проскальзывания и горизонтальной плоскости, независимо от конкретного значения $\tau$.

\subsection{Параметры эксперимента}

\begin{table}[H]
    \centering
    \begin{tabular}{|l|l|}
        \hline
        Параметр                      & Значение                 \\
        \hline
        Число траекторий $K$          & 512                      \\
        Горизонт $T$                  & 30                       \\
        Температура $\lambda$         & 1.0                      \\
        Шаг интегрирования $\Delta t$ & 0.05 с                   \\
        $K_p$ (плоскость)             & 100                      \\
        $K_d$ (плоскость)             & $2\sqrt{100} \approx 20$ \\
        $K_d$ (проскальзывание)       & $2\sqrt{100} \approx 20$ \\
        Радиус колёс $r$              & 0.08 м                   \\
        \hline
    \end{tabular}
    \caption{Параметры эксперимента с колёсным роботом}
    \label{tab:boxer_params}
\end{table}

\subsection{Результаты}

% TODO: провести эксперимент point-to-point навигации
% TODO: записать траекторию робота, построить графики позиции
% TODO: построить график невязки ограничения \eqref{eq:no_slip} по времени
% TODO: сравнить с классическим MPPI (штрафной метод за нарушение ограничения)
% TODO: измерить время вычисления одного шага управления (на GPU)

\section{Эксперимент 2: Dual-arm YuMi с жёстким объектом}

\subsection{Описание системы}

Рассматривается двурукий манипулятор ABB IRB 14000 (YuMi), состоящий из двух 7-DOF рук. Каждая рука имеет 7 вращательных и 2 призматических (пальцы) сочленения; в эксперименте пальцы фиксированы, итого $n_v = 14$ управляемых степеней свободы.

Два манипулятора удерживают жёсткий объект (стержень), образуя кинематически замкнутую цепь. Объект фиксирован в схватах обоих манипуляторов, что создаёт голономное ограничение на 6 степеней свободы (3 позиционных + 3 ориентационных).

\subsection{Формулировка ограничения}

Пусть $\mathcal{E}_1$, $\mathcal{E}_2$ --- системы координат конечных эффекторов левой и правой руки, $T_{\mathcal{W}}^{\mathcal{E}_i} \in SE(3)$ --- их позы в мировой системе координат, $T_A^B$ --- постоянное преобразование между точками крепления на жёстком объекте. Голономное ограничение:
\begin{equation}
    \varphi(q) = \mathbf{e}_{SE(3)}\!\left(T_{\mathcal{W}}^{\mathcal{E}_1} \cdot T_A^B, \; T_{\mathcal{W}}^{\mathcal{E}_2}\right) = 0,
    \label{eq:yumi_holonomic}
\end{equation}
где функция ошибки $\mathbf{e}_{SE(3)} : SE(3) \times SE(3) \to \mathbb{R}^6$ определена как:
\begin{equation}
    \mathbf{e}_{SE(3)}(T_1, T_2) = \begin{bmatrix} \mathbf{p}_2 - \mathbf{p}_1 \\ \log(R_1^T R_2) \end{bmatrix},
    \label{eq:se3_error}
\end{equation}
$\log : SO(3) \to \mathbb{R}^3$ --- матричный логарифм.

Для приведения к форме Пфаффа $A\dot{v} = b$ используется стабилизированное кинематическое дифференциальное уравнение:
\begin{equation}
    \begin{bmatrix}
        \dot{v}_{E_2} - \dot{v}_B \\
        \dot{\omega}_{E_2}^{\mathcal{E}_2} - \dot{\omega}_B^{\mathcal{E}_2}
    \end{bmatrix}
    + K_d
    \begin{bmatrix}
        v_{E_2} - v_B \\
        \omega_{E_2}^{\mathcal{E}_2} - \omega_B^{\mathcal{E}_2}
    \end{bmatrix}
    + K_p \cdot \mathbf{e}_{SE(3)}(\ldots) = 0,
    \label{eq:yumi_stab}
\end{equation}
где $v_B$, $\omega_B$ --- линейная и угловая скорости точки $B$ жёсткого объекта, вычисляемые через кинематику твёрдого тела:
\begin{equation}
    v_B = v_{E_1} + \omega_{E_1} \times r_{AB}, \quad \omega_B = \omega_{E_1}.
\end{equation}

Матрица ограничений и правая часть выражаются через якобианы конечных эффекторов:
\begin{equation}
    \begin{aligned}
        A & = \begin{bmatrix}
                  \tilde{J}_{d,v} \\
                  \tilde{J}_{d,\omega}
              \end{bmatrix},                     \\
        b & = -\begin{bmatrix}
                   \dot{\tilde{J}}_{d,v} \\
                   \dot{\tilde{J}}_{d,\omega}
               \end{bmatrix} v
        - K_d \begin{bmatrix}
                  v_d \\
                  \omega_d^{\mathcal{E}_2}
              \end{bmatrix}
        - K_p \cdot \mathbf{e}_{SE(3)}(\ldots),
    \end{aligned}
    \label{eq:yumi_Ab}
\end{equation}
где $\tilde{J}_{d,v} = J_{E_2,v} - J_{E_1,v} + [\hat{r}_{AB}] J_{E_1,\omega}$, $\tilde{J}_{d,\omega} = R_{E_2}^T (J_{E_2,\omega} - J_{E_1,\omega})$, $[\hat{r}_{AB}]$ --- оператор векторного произведения.

Устойчивость стабилизации \eqref{eq:yumi_stab} по позиционной компоненте следует из линейной теории ($K_p = \omega^2$, $K_d = 2\omega$ --- критическое демпфирование). Устойчивость по ориентационной компоненте доказывается через функцию Ляпунова $V = \frac{1}{2}\tilde{\omega}^T\tilde{\omega} + \frac{K_p}{2}\mathbf{e}^T\mathbf{e}$ с использованием свойства $\mathbf{e}^T J_r^{-1}(\mathbf{e}) = \mathbf{e}^T$ матричного логарифма~\cite{yumi_constraint_ref}.

\subsection{Задача управления}

Один из конечных эффекторов (левая рука, фрейм $\mathcal{E}_1$) должен следовать заданной траектории $T_{\mathcal{W}}^{\mathcal{D}}(t)$ при сохранении жёсткой связи между руками. Управление задаётся через ограничения: ограничение жёсткой связи обеспечивает замкнутость кинематической цепи, а ограничение следования траектории задаёт желаемое движение.

В контексте Gauss-MPPI ограничение жёсткой связи \eqref{eq:yumi_Ab} включается в проекцию Гаусса, а MPPI оптимизирует функцию стоимости, учитывающую отклонение от желаемой траектории и управляющие моменты.

\subsection{Параметры эксперимента}

\begin{table}[H]
    \centering
    \begin{tabular}{|l|l|}
        \hline
        Параметр                & Значение                       \\
        \hline
        DOF системы             & 14 (7 + 7, пальцы фиксированы) \\
        Размерность ограничения & 6 (3 позиция + 3 ориентация)   \\
        Фрейм $\mathcal{E}_1$   & \texttt{yumi\_link\_7\_l}      \\
        Фрейм $\mathcal{E}_2$   & \texttt{yumi\_link\_7\_r}      \\
        $K_p$ (ограничение)     & 1000                           \\
        $K_d$ (ограничение)     & $2\sqrt{1000} \approx 63$      \\
        \hline
    \end{tabular}
    \caption{Параметры эксперимента с Dual-arm YuMi}
    \label{tab:yumi_params}
\end{table}

% TODO: интегрировать URDF модель YuMi в MuJoCo/MJX
% TODO: реализовать ограничение жёсткой связи \eqref{eq:yumi_Ab} как класс Constraint
% TODO: реализовать вычисление SE(3) ошибки и матричного логарифма в JAX
% TODO: задать желаемую траекторию (круговое движение в вертикальной плоскости)
% TODO: провести эксперимент, записать поведение фреймов $\mathcal{E}_1$, $\mathcal{E}_2$
% TODO: построить графики: ошибка ограничения $\mathbf{e}_{SE(3)}$ по времени
% TODO: построить графики: ошибка слежения за траекторией по времени
% TODO: измерить время вычисления одного шага (rollout + проекция)

\subsection{Результаты}

% TODO: представить результаты экспериментов
% TODO: оценить точность выполнения ограничения жёсткой связи
% TODO: оценить качество слежения за траекторией
