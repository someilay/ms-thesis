\newpage
\begin{center}
  \textbf{\large 3. ПРОГРАММНАЯ РЕАЛИЗАЦИЯ И ЭКСПЕРИМЕНТЫ}
\end{center}
\refstepcounter{chapter}
\addcontentsline{toc}{chapter}{3. ПРОГРАММНАЯ РЕАЛИЗАЦИЯ И ЭКСПЕРИМЕНТЫ}

\section{Реализация классического MPPI}

Выполнена программная реализация классического алгоритма MPPI в среде симуляции MuJoCo. Реализация включает:
\begin{itemize}
    \item Параллельную выборку траекторий на CPU;
    \item Интеграцию с физическим движком MuJoCo для моделирования динамики;
    \item Тестирование на стандартных задачах управления (CartPole, Pendulum и др.).
\end{itemize}

% TODO: реализовать параллельную версию на GPU (PyTorch/CUDA)
% TODO: добавить визуализацию результатов
% TODO: добавить Винеровский процесс в схему интеграции MuJoCo

\section{Интегрирование СДУ}

Проведены эксперименты по численному интегрированию физических систем, представленных в виде стохастических дифференциальных уравнений:
\begin{itemize}
    \item Реализованы методы Эйлера--Маруямы и детерминированное приближение;
    \item Исследовано влияние шага интегрирования на точность и устойчивость;
    \item Выявлены особенности работы с механическими системами (проверка сохранения энергии).
\end{itemize}

\section{Реализация алгоритма Gauss-MPPI}

% TODO: реализовать эффективный QP-решатель для проекции по Гауссу, совместимый с GPU
% TODO: интегрировать Gauss-MPPI с симулятором MuJoCo
% TODO: оптимизировать производительность для работы в реальном времени

\section{Экспериментальные сценарии}

Для валидации предложенного метода выбраны следующие экспериментальные сценарии.

\subsection{Неголономный мобильный робот}

Колёсный робот с ограничением отсутствия проскальзывания:
\begin{equation}
    \dot{x}\sin\theta - \dot{y}\cos\theta = 0.
\end{equation}

% TODO: реализовать модель неголономного робота
% TODO: провести эксперименты, сравнить точность выполнения ограничения

\subsection{Робот-манипулятор с ограничениями в операционном пространстве}

Задача следования по траектории с ограничениями на конечный эффектор:
\begin{equation}
    J(q)\dot{v} + \dot{J}v = \ddot{x}_d.
\end{equation}

% TODO: реализовать модель манипулятора
% TODO: задать ограничения в операционном пространстве
% TODO: провести эксперименты

\subsection{Система с контактными взаимодействиями}

% TODO: реализовать модель с контактными ограничениями
% TODO: проверить работу метода при наличии ограничений трения
