\newpage
\begin{center}
  \textbf{\large 2. МЕТОДОЛОГИЯ}
\end{center}
\refstepcounter{chapter}
\addcontentsline{toc}{chapter}{2. МЕТОДОЛОГИЯ}

\section{Численное интегрирование стохастических дифференциальных уравнений}

Ключевым этапом алгоритма MPPI является моделирование траекторий системы, описываемой стохастическим дифференциальным уравнением (СДУ):
\begin{equation}
    dx = f(x, u)\,dt + B(x)\,dW,
\end{equation}
где $dW$ --- приращение Винеровского процесса.

Качество численных методов для СДУ характеризуется двумя типами сходимости:
\begin{itemize}
    \item \textbf{Сильная сходимость} (порядок $\gamma$) --- сходимость по траекториям:
    \begin{equation}
        \mathbb{E}\left[|x_N - x(T)|^2\right]^{1/2} \leq C \cdot \Delta t^\gamma.
    \end{equation}
    Важна, когда требуется точное воспроизведение отдельных реализаций процесса.
    
    \item \textbf{Слабая сходимость} (порядок $\beta$) --- сходимость по распределениям:
    \begin{equation}
        \left|\mathbb{E}[g(x_N)] - \mathbb{E}[g(x(T))]\right| \leq C \cdot \Delta t^\beta
    \end{equation}
    для гладких функций $g$. Важна для вычисления математических ожиданий (как в MPPI).
\end{itemize}

\subsection{Метод Эйлера--Маруямы}

Простейшая схема численного интегрирования СДУ:
\begin{equation}
    x_{k+1} = x_k + f(x_k, u_k)\,\Delta t + B(x_k)\,\Delta W_k,
\end{equation}
где $\Delta W_k \sim \mathcal{N}(0, \Delta t \cdot I)$ --- дискретное приращение Винеровского процесса. Метод имеет сильный порядок сходимости $0.5$ и слабый порядок $1.0$.

\subsection{Метод Мильштейна}

Схема более высокого порядка, учитывающая поправку Ито:
\begin{equation}
    x_{k+1} = x_k + f(x_k, u_k)\,\Delta t + B(x_k)\,\Delta W_k + \frac{1}{2}B(x_k)\frac{\partial B}{\partial x}(x_k)\left[(\Delta W_k)^2 - \Delta t\right].
\end{equation}
Метод имеет сильный порядок сходимости $1.0$, но требует вычисления производных матрицы диффузии.

\subsection{Детерминированное приближение}

В практических реализациях MPPI часто используется детерминированная модель с аддитивным шумом в управлении:
\begin{equation}
    x_{k+1} = x_k + f(x_k, u_k + \epsilon_k)\,\Delta t, \quad \epsilon_k \sim \mathcal{N}(0, \Sigma),
\end{equation}
что позволяет применять стандартные методы интегрирования ОДУ (Эйлера, Рунге--Кутты) и упрощает реализацию на GPU.

\subsection{Симплектические интеграторы}

Механические системы естественно описываются в гамильтоновой форме:
\begin{equation}
    \dot{q} = \frac{\partial H}{\partial p}, \quad \dot{p} = -\frac{\partial H}{\partial q},
\end{equation}
где $q$ --- обобщённые координаты, $p$ --- обобщённые импульсы, $H(q, p)$ --- гамильтониан. Симплектические интеграторы (Штёрмера--Верле, RATTLE) сохраняют геометрическую структуру фазового потока, обеспечивая:
\begin{itemize}
    \item Ограниченность ошибки энергии на бесконечном интервале;
    \item Качественно верное поведение траекторий;
    \item Долговременную устойчивость численного решения.
\end{itemize}

\section{Алгоритм Gauss-MPPI}

Предлагаемый метод Gauss-MPPI интегрирует принцип наименьшего принуждения Гаусса в процедуру семплирования траекторий MPPI. На каждом шаге интегрирования свободное ускорение проецируется на множество допустимых ускорений с помощью уравнения Удвадия--Калабы.

Для систем с Пфаффовыми ограничениями $A(q, v, t)\dot{v} = b(q, v, t)$ проекция свободного ускорения $\dot{v}_{\text{free}} = M^{-1}(Q + Bu)$ на множество допустимых ускорений имеет вид:
\begin{equation}
    \dot{v} = \dot{v}_{\text{free}} + M^{-1/2}(AM^{-1/2})^{+}(b - A\dot{v}_{\text{free}}).
\end{equation}

Данная проекция выполняется для каждой семплированной траектории на каждом временном шаге, что гарантирует выполнение ограничений во всех точках горизонта планирования.

% TODO: формализовать применение принципа Гаусса для стохастических систем
% TODO: определить условия корректности и границы применимости
% TODO: исследовать свойства сходимости модифицированного алгоритма

\section{Применимость принципа Гаусса к СДУ}

Классический принцип Гаусса сформулирован для детерминированных систем. Применение принципа к стохастическим системам требует дополнительного исследования:
\begin{itemize}
    \item Открытый вопрос: как корректно определить <<свободное ускорение>> при наличии стохастического члена $B(x)\,dW$;
    \item Рассматриваются подходы: проекция на каждом шаге интегрирования, модификация функции принуждения с учётом шума.
\end{itemize}

% TODO: формализовать и доказать корректность применения принципа Гаусса для СДУ
% TODO: рассмотреть альтернативные определения свободного ускорения для стохастических систем

\section{Вычислительная сложность}

% TODO: оценить вычислительную сложность решения QP-задачи на каждом шаге
% TODO: оценить накладные расходы Gauss-MPPI по сравнению с классическим MPPI
% TODO: проанализировать возможность батчевого решения QP для всех траекторий на GPU
