\documentclass[12pt,a4paper]{article}

% Кодировка и язык
\usepackage[utf8]{inputenc}
\usepackage[T2A]{fontenc}
\usepackage[russian]{babel}

% Математика
\usepackage{amsmath,amssymb,amsthm}

% Поля
\usepackage[left=2.5cm,right=2cm,top=2cm,bottom=2cm]{geometry}

% Интервал
\usepackage{setspace}
\onehalfspacing

% Заголовки
\usepackage{titlesec}
\titleformat{\section}{\large\bfseries}{\thesection.}{0.5em}{}
\titleformat{\subsection}{\normalsize\bfseries}{\thesubsection.}{0.5em}{}

% Списки
\usepackage{enumitem}

% Библиография
\usepackage{cite}

% Алгоритмы
\usepackage{algorithm}
\usepackage{algpseudocode}
\floatname{algorithm}{Алгоритм}
\renewcommand{\algorithmicrequire}{\textbf{Вход:}}
\renewcommand{\algorithmicensure}{\textbf{Выход:}}

\begin{document}

% ============================================================================
%                               ЗАГОЛОВОК
% ============================================================================

\begin{center}
    \textbf{\large Тема:}
    
    \vspace{0.3cm}
    
    {\Large \textbf{Управление методом Model Predictive Path Integral для систем с линейными по ускорениям ограничениями}}
    
    \vspace{0.3cm}
    
    {\large \textit{Model Predictive Path Integral Control for Systems with Acceleration-Level Linear Constraints}}
\end{center}

\vspace{1cm}

% ============================================================================
%                                 ЦЕЛЬ
% ============================================================================

\section{Цель работы}

Целью данной работы является разработка и исследование метода управления на основе Model Predictive Path Integral (MPPI), обеспечивающего \textbf{точное выполнение линейных по ускорениям ограничений} вида (Пфаффовы ограничения):
\begin{equation}
    A(q, v, t) \, \dot{v} = b(q, v, t),
\end{equation}
где $q \in \mathbb{R}^n$ — обобщённые координаты, $v$ — обобщённые скорости, $A \in \mathbb{R}^{m \times n}$ — матрица ограничений, $b \in \mathbb{R}^m$ — вектор правой части.

Предлагаемый подход основан на интеграции \textbf{принципа наименьшего принуждения Гаусса} в процедуру семплирования траекторий MPPI, что позволяет гарантировать выполнение ограничений на каждом шаге интегрирования динамики системы.

% ============================================================================
%                                ЗАДАЧИ
% ============================================================================

\section{Задачи исследования}

Для достижения поставленной цели необходимо решить следующие задачи:

\begin{enumerate}[leftmargin=2cm, label=\textbf{\arabic*.}]
    
    \item \textbf{Анализ существующих методов}
    \begin{itemize}[leftmargin=0.5cm]
        \item Провести обзор методов обработки ограничений в MPPI (барьерные функции, штрафные методы, вероятностные подходы).
        \item Выявить преимущества и недостатки существующих подходов.
    \end{itemize}
    
    \item \textbf{Разработка теоретических основ метода Gauss-MPPI}
    \begin{itemize}[leftmargin=0.5cm]
        \item Сформулировать задачу проекции свободного ускорения на множество допустимых ускорений как задачу квадратичного программирования (QP).
        \item Исследовать применимость метода наименьшего принуждения Гаусса для стохастических дифференциальных уравнений.
        \item Исследовать свойства сходимости модифицированного алгоритма MPPI.
    \end{itemize}
    
    \item \textbf{Разработка алгоритма и его программная реализация}
    \begin{itemize}[leftmargin=0.5cm]
        \item Разработать алгоритм Gauss-MPPI с интегрированным QP-решателем.
        \item Реализовать параллельную версию алгоритма для GPU (Pytorch/CUDA).
        \item Оценить вычислительную сложность и возможность работы в реальном времени.
    \end{itemize}
    
    \item \textbf{Экспериментальная проверка на модельных задачах}
    \begin{itemize}[leftmargin=0.5cm]
        \item Неголономный мобильный робот (ограничение отсутствия проскальзывания).
        \item Робот-манипулятор с ограничениями в операционном пространстве.
        \item Система с контактными взаимодействиями.
    \end{itemize}
    
    \item \textbf{Сравнительный анализ}
    \begin{itemize}[leftmargin=0.5cm]
        \item Сравнить предложенный метод с классическим MPPI, CBF-MPPI \\ и штрафными методами.
        \item Оценить точность выполнения ограничений, качество управления и вычислительные затраты.
    \end{itemize}
    
\end{enumerate}

% ============================================================================
%                           ОЖИДАЕМЫЕ РЕЗУЛЬТАТЫ
% ============================================================================

\section{Ожидаемые результаты}

\begin{enumerate}[leftmargin=2cm, label=\textbf{\arabic*.}]
    \item Теоретическое обоснование метода Gauss-MPPI для систем с Пфаффовыми ограничениями.
    \item Программная реализация алгоритма с поддержкой параллельных вычислений.
    \item Экспериментальное подтверждение эффективности метода на задачах робототехники.
\end{enumerate}

% ============================================================================
%                           ЛИТЕРАТУРНЫЙ ОБЗОР
% ============================================================================

\section{Литературный обзор}

\subsection{Теоретические основы Path Integral Control}

Теория управления на основе интегралов по траекториям берёт начало в работах Каппена~\cite{kappen2005path}, который показал связь между стохастическим оптимальным управлением и статистической механикой. Для определённого класса задач (линейная зависимость динамики от управления, квадратичная стоимость управления) уравнение Гамильтона--Якоби--Беллмана преобразуется в линейное уравнение, решение которого выражается через интеграл по траекториям:
\begin{equation}
    u^*(x, t) = \mathbb{E}\left[ u(x, t) \cdot \exp\left(-\frac{1}{\lambda} S(\tau)\right) \right],
\end{equation}
где $S(\tau)$ — стоимость траектории $\tau$, $\lambda$ — параметр температуры.

Теодору и др.~\cite{theodorou2010generalized} обобщили этот подход на более широкий класс систем и предложили алгоритм PI$^2$ (Policy Improvement with Path Integrals) для обучения с подкреплением.

\subsection{Model Predictive Path Integral Control}

Метод MPPI был предложен Уильямсом и др.~\cite{williams2017mppi} как подход к модельно-предиктивному управлению, основанный на стохастической выборке траекторий. В отличие от градиентных методов, MPPI генерирует множество случайных траекторий и вычисляет оптимальное управление путём взвешенного усреднения:
\begin{equation}
    u^* = \sum_{k=1}^{K} w_k \cdot u_k, \quad w_k = \frac{\exp(-\frac{1}{\lambda} S_k)}{\sum_{j=1}^{K} \exp(-\frac{1}{\lambda} S_j)},
\end{equation}
где $K$ — число траекторий, $S_k$ — стоимость $k$-й траектории.

Ключевые преимущества MPPI:
\begin{itemize}
    \item Не требует вычисления градиентов функции стоимости;
    \item Работает с невыпуклыми и негладкими функциями стоимости;
    \item Эффективно параллелизуется на GPU~\cite{williams2017aggressive}.
\end{itemize}

MPPI успешно применяется в автономном вождении~\cite{williams2017aggressive}, управлении квадрокоптерами и манипуляторами.

\subsection{Постановка задачи и алгоритм MPPI}

MPPI решает задачу стохастического оптимального управления для систем вида:
\begin{equation}
\begin{aligned}
    & dx = f(x, u) dt + B(x) dw \\
    & x(t_0) = x_0
\end{aligned}
\end{equation}
где $x \in \mathbb{R}^n$ — вектор состояния, $u \in \mathbb{R}^m$ — вектор управления, $f$ — функция динамики, $B$ — матрица диффузии, $dw$ — Винеровский процесс.

Цель — минимизация функционала стоимости на горизонте $T$:
\begin{equation}
    J(u) = \phi(x_T) + \int_{t_0}^{t_0 + T} \mathcal{L}(x_t, u_t)\,dt,
\end{equation}
где $\phi(x_T)$ — терминальная стоимость, $\mathcal{L}(x, u)$ — мгновенная стоимость.

\begin{algorithm}[H]
\caption{Model Predictive Path Integral (MPPI)}
\label{alg:mppi}
\begin{algorithmic}[1]
\Require $x_0$ — начальное состояние, $\bar{U} = \{\bar{u}_0, \ldots, \bar{u}_{N-1}\}$ — номинальное управление
\Ensure $u^*$ — оптимальное управление
\Loop
    \For{$k = 1, \ldots, K$} \Comment{Параллельно на GPU}
        \State $x_0^{(k)} \gets x_0$
        \For{$t = 0, \ldots, N-1$}
            \State $\epsilon_t^{(k)} \sim \mathcal{N}(0, \Sigma)$ \Comment{Выборка возмущений}
            \State $u_t^{(k)} \gets \bar{u}_t + \epsilon_t^{(k)}$
            \State $x_{t+1}^{(k)} \gets x_t^{(k)} + f(x_t^{(k)}, u_t^{(k)}) \Delta t$ \Comment{Интегрирование}
        \EndFor
        \State $S^{(k)} \gets \phi(x_N^{(k)}) + \sum_{t=0}^{N-1} \mathcal{L}(x_t^{(k)}, u_t^{(k)}) \Delta t$ \Comment{Стоимость}
    \EndFor
    \For{$k = 1, \ldots, K$}
        \State $w^{(k)} \gets \exp\left(-\frac{1}{\lambda} S^{(k)}\right) \Big/ \sum_{j=1}^{K} \exp\left(-\frac{1}{\lambda} S^{(j)}\right)$ \Comment{Веса}
    \EndFor
    \For{$t = 0, \ldots, N-1$}
        \State $\bar{u}_t \gets \bar{u}_t + \sum_{k=1}^{K} w^{(k)} \epsilon_t^{(k)}$ \Comment{Обновление}
    \EndFor
    \State Применить $\bar{u}_0$ к системе
    \State Сдвинуть горизонт: $\bar{U} \gets \{\bar{u}_1, \ldots, \bar{u}_{N-1}, \bar{u}_{N-1}\}$
\EndLoop
\end{algorithmic}
\end{algorithm}

Параметр $\lambda > 0$ регулирует «температуру» распределения: при $\lambda \to 0$ алгоритм выбирает траекторию с минимальной стоимостью, при $\lambda \to \infty$ — равномерное усреднение.

\subsection{Обработка ограничений в MPPI}

Исходная формулировка MPPI не предусматривает явной обработки ограничений. Существующие подходы можно классифицировать следующим образом:

\textbf{1. Штрафные методы.} Наиболее простой подход — добавление штрафов за нарушение ограничений в функцию стоимости~\cite{williams2017mppi}. Недостатки: не гарантируют выполнение ограничений, требуют настройки весовых коэффициентов.

\textbf{2. Методы на основе барьерных функций.} Shield-MPPI~\cite{shieldmppi2024} использует Control Barrier Functions (CBF) для фильтрации небезопасных управлений. GS-MPPI~\cite{gsmppi2024} применяет композитные CBF для систем с множественными ограничениями. Эти методы гарантируют выполнение ограничений типа неравенств, но увеличивают вычислительную сложность.

\textbf{3. Вероятностные методы.} SC-MPPI~\cite{scmppi2023} встраивает фильтр безопасности в процесс выборки траекторий. BC-MPPI~\cite{bcmppi2024} присваивает вероятность допустимости каждой траектории, и корректирует веса соответственно.

\textbf{4. Проекционные методы.} $\pi$-MPPI~\cite{pimppi2024} использует проекцию для обеспечения ограничений на величину и производные управления.

\subsection{Принцип наименьшего принуждения Гаусса}

Принцип наименьшего принуждения Гаусса утверждает: из всех кинематически возможных движений системы реализуется то, для которого величина принуждения минимальна. Математически принцип формулируется как задача квадратичного программирования:
\begin{equation}
\begin{aligned}
    & \min_{\dot{v}} && [\dot{v} - a]^T M [\dot{v} - a] \\
    & \text{s.t.} && A\dot{v} = b, \\
    & && a = M^{-1}Q,
\end{aligned}
\end{equation}
где $M$ — матрица масс, $Q$ — вектор обобщённых сил, $a = M^{-1}Q$ — ускорение системы без ограничений (свободное ускорение).

\subsection{Уравнение Удвадия--Калабы}

Удвадия и Калаба~\cite{udwadia1992, udwadia2002} разработали универсальный метод получения уравнений движения для систем с ограничениями вида $A(q, v, t)\dot{v} = b(q, v, t)$. Решение задачи минимизации принуждения имеет аналитический вид:
\begin{equation}
    \dot{v} = \dot{v}_{\text{free}} + M^{-1/2}(AM^{-1/2})^{+}(b - A\dot{v}_{\text{free}}),
\end{equation}
где $(\cdot)^+$ — псевдообратная матрица Мура--Пенроуза.

Уравнение применимо к:
\begin{itemize}
    \item Голономным ограничениям: $\phi(q, t) = 0$;
    \item Неголономным ограничениям: $\phi(q, v, t) = 0$;
    \item Ограничениям в операционном пространстве: $J(q)\dot{v} + \dot{J}v = \ddot{x}_d$.
\end{itemize}

\subsection{Выводы из обзора}

Анализ литературы позволяет сделать следующие выводы:

\begin{enumerate}
    \item Существующие методы обработки ограничений в MPPI ориентированы преимущественно на \textbf{ограничения типа неравенств} (безопасность, границы управления).
    
    \item \textbf{Ограничения равенств на уровне ускорений} (Пфаффовы ограничения) в контексте MPPI практически не исследованы.
    
    \item Принцип наименьшего принуждения Гаусса и уравнение Удвадия--Калабы предоставляют \textbf{аналитическое решение} для проекции на множество допустимых ускорений.
    
    \item \textbf{Связь между MPPI и принципом Гаусса не исследована}.
\end{enumerate}

% ============================================================================
%                    ЧИСЛЕННОЕ ИНТЕГРИРОВАНИЕ СДУ
% ============================================================================

\section{Численное интегрирование стохастических \\ дифференциальных уравнений}

Ключевым этапом алгоритма MPPI является моделирование траекторий системы, описываемой стохастическим дифференциальным уравнением (СДУ):
\begin{equation}
    dx = f(x, u)\,dt + B(x)\,dW,
\end{equation}
где $dW$ — приращение Винеровского процесса.

Качество численных методов для СДУ характеризуется двумя типами сходимости:
\begin{itemize}
    \item \textbf{Сильная сходимость} (порядок $\gamma$) — сходимость по траекториям:
    \begin{equation}
        \mathbb{E}\left[|x_N - x(T)|^2\right]^{1/2} \leq C \cdot \Delta t^\gamma.
    \end{equation}
    Важна, когда требуется точное воспроизведение отдельных реализаций процесса.
    
    \item \textbf{Слабая сходимость} (порядок $\beta$) — сходимость по распределениям:
    \begin{equation}
        \left|\mathbb{E}[g(x_N)] - \mathbb{E}[g(x(T))]\right| \leq C \cdot \Delta t^\beta
    \end{equation}
    для гладких функций $g$. Важна для вычисления математических ожиданий (как в MPPI).
\end{itemize}

\subsection{Метод Эйлера--Маруямы}

Простейшая схема численного интегрирования СДУ:
\begin{equation}
    x_{k+1} = x_k + f(x_k, u_k)\,\Delta t + B(x_k)\,\Delta W_k,
\end{equation}
где $\Delta W_k \sim \mathcal{N}(0, \Delta t \cdot I)$ — дискретное приращение Винеровского процесса. Метод имеет сильный порядок сходимости $0.5$ и слабый порядок $1.0$.

\subsection{Метод Мильштейна}

Схема более высокого порядка, учитывающая поправку Ито:
\begin{equation}
    x_{k+1} = x_k + f(x_k, u_k)\,\Delta t + B(x_k)\,\Delta W_k + \frac{1}{2}B(x_k)\frac{\partial B}{\partial x}(x_k)\left[(\Delta W_k)^2 - \Delta t\right].
\end{equation}
Метод имеет сильный порядок сходимости $1.0$, но требует вычисления производных матрицы диффузии.

\subsection{Детерминированное приближение}

В практических реализациях MPPI часто используется детерминированная модель с аддитивным шумом в управлении:
\begin{equation}
    x_{k+1} = x_k + f(x_k, u_k + \epsilon_k)\,\Delta t, \quad \epsilon_k \sim \mathcal{N}(0, \Sigma),
\end{equation}
что позволяет применять стандартные методы интегрирования ОДУ (Эйлера, Рунге--Кутты) и упрощает реализацию на GPU.

\subsection{Симплектические интеграторы}

Механические системы естественно описываются в гамильтоновой форме:
\begin{equation}
    \dot{q} = \frac{\partial H}{\partial p}, \quad \dot{p} = -\frac{\partial H}{\partial q},
\end{equation}
где $q$ — обобщённые координаты, $p$ — обобщённые импульсы, $H(q, p)$ — гамильтониан (полная энергия системы). Такие системы обладают важным свойством: фазовый поток сохраняет симплектическую структуру (объём в фазовом пространстве).

Стандартные методы (Эйлер, Рунге--Кутта) не сохраняют это свойство, что приводит к накоплению ошибки в энергии на длительных интервалах времени. Симплектические интеграторы (Штёрмера--Верле, RATTLE) сохраняют геометрическую структуру фазового потока, обеспечивая:
\begin{itemize}
    \item Ограниченность ошибки энергии на бесконечном интервале;
    \item Качественно верное поведение траекторий;
    \item Долговременную устойчивость численного решения.
\end{itemize}

% ============================================================================
%                          СДЕЛАНО НА ДАННЫЙ МОМЕНТ
% ============================================================================

\section{Сделано на данный момент}

\subsection{Реализация классического MPPI}

Выполнена программная реализация классического алгоритма MPPI в среде симуляции MuJoCo. Реализация включает:
\begin{itemize}
    \item Параллельную выборку траекторий на CPU (медленно);
    \item Интеграцию с физическим движком MuJoCo для моделирования динамики; Пока проблемно внедрить Винеровский процесс в схему интеграции
    \item Тестирование на стандартных задачах управления (CartPole, Pendulum и др.) пока без визуализации.
\end{itemize}

\subsection{Алгоритм Gauss-MPPI}

Приблизительный алгоритм интеграции принципа наименьшего принуждения Гаусса в MPPI для систем с Пфаффовыми ограничениями:

\begin{algorithm}[H]
\caption{Gauss-MPPI (предварительная версия)}
\label{alg:gauss-mppi}
\begin{algorithmic}[1]
\Require $x_0$, $\bar{U}$, ограничения $A(q,v,t)\dot{v} = b(q,v,t)$
\For{$k = 1, \ldots, K$}
    \State $x_0^{(k)} \gets x_0$
    \For{$t = 0, \ldots, N-1$}
        \State $\epsilon_t^{(k)} \sim \mathcal{N}(0, \Sigma)$
        \State $u_t^{(k)} \gets \bar{u}_t + \epsilon_t^{(k)}$
        \State $\dot{v}_{\text{free}} \gets M^{-1}(Q + Bu_t^{(k)})$ \Comment{Свободное ускорение}
        \State $\dot{v}^{(k)} \gets \textsc{GaussProject}(\dot{v}_{\text{free}}, A, b, M)$ \Comment{Проекция}
        \State $x_{t+1}^{(k)} \gets \textsc{Integrate}(x_t^{(k)}, \dot{v}^{(k)}, \Delta t)$
    \EndFor
    \State Вычислить $S^{(k)}$, $w^{(k)}$
\EndFor
\State Обновить $\bar{U}$
\end{algorithmic}
\end{algorithm}

\subsection{Интегрирование СДУ}

Проведены эксперименты по численному интегрированию физических систем, представленных в виде стохастических дифференциальных уравнений:
\begin{itemize}
    \item Реализованы методы Эйлера--Маруямы и детерминированное приближение;
    \item Исследовано влияние шага интегрирования на точность и устойчивость;
    \item Выявлены особенности работы с механическими системами (проверка сохранения энергии).
\end{itemize}

\subsection{Применимость принципа Гаусса к СДУ}

Исследуется вопрос корректности применения принципа наименьшего принуждения Гаусса для стохастических систем:
\begin{itemize}
    \item Классический принцип Гаусса сформулирован для детерминированных систем;
    \item Открытый вопрос: как корректно определить «свободное ускорение» при наличии стохастического члена $B(x)\,dW$;
    \item Рассматриваются подходы: проекция на каждом шаге интегрирования, модификация функции принуждения с учётом шума.
\end{itemize}

% ============================================================================
%                           ДАЛЬНЕЙШИЕ ШАГИ
% ============================================================================

\section{Дальнейшие шаги}

\subsection{Теоретическое обоснование}

\begin{enumerate}
    \item Формализовать применение принципа Гаусса для стохастических систем: определить условия корректности и границы применимости.
    \item Исследовать свойства сходимости алгоритма Gauss-MPPI: показать, что модификация не нарушает сходимость к оптимальному управлению.
    \item Проанализировать вычислительную сложность: оценить накладные расходы на решение QP-задачи на каждом шаге.
\end{enumerate}

\subsection{Программная реализация}

\begin{enumerate}
    \item Реализовать эффективный QP-решатель для проекции по Гауссу, совместимый с GPU (батчевое решение для всех траекторий).
    \item Интегрировать Gauss-MPPI с симуляторами MuJoCo или Gazebo.
    \item Оптимизировать производительность для работы в реальном времени.
\end{enumerate}

\subsection{Экспериментальная валидация}

\begin{enumerate}
    \item \textbf{Неголономный робот:} колёсный робот с ограничением отсутствия проскальзывания — проверить точность выполнения ограничения $\dot{x}\sin\theta - \dot{y}\cos\theta = 0$.
    \item \textbf{Манипулятор:} робот-манипулятор с ограничениями в операционном пространстве — задача следования по траектории с ограничениями на конечный эффектор.
    \item \textbf{Контактные задачи:} система с контактными взаимодействиями — проверить работу метода при наличии ограничений трения.
\end{enumerate}

\subsection{Сравнительный анализ}

\begin{enumerate}
    \item Сравнить Gauss-MPPI с классическим MPPI по метрикам: точность выполнения ограничений, качество управления (стоимость), время вычислений.
    \item Сравнить с альтернативными методами: CBF-MPPI, штрафные методы, $\pi$-MPPI.
    \item Исследовать чувствительность к параметрам: число траекторий $K$, температура $\lambda$, шаг интегрирования $\Delta t$.
\end{enumerate}

\subsection{Структурирование работы в единый репозиторий}

\begin{enumerate}
    \item Выложить текущие разработки (python notebooks) в github
    \item Собрать все теоретические наработки в структурированный документ
    \item Начать оформлять работу согласно шаблону магистерской работы
\end{enumerate}

% ============================================================================
%                              БИБЛИОГРАФИЯ
% ============================================================================

\bibliographystyle{unsrt}
\bibliography{ref}

\end{document}

