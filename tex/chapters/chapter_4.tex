\newpage
\begin{center}
  \textbf{\large 4. РЕЗУЛЬТАТЫ И СРАВНИТЕЛЬНЫЙ АНАЛИЗ}
\end{center}
\refstepcounter{chapter}
\addcontentsline{toc}{chapter}{4. РЕЗУЛЬТАТЫ И СРАВНИТЕЛЬНЫЙ АНАЛИЗ}

\section{Метрики оценки}

Для сравнительного анализа предлагаемого метода используются следующие метрики:
\begin{itemize}
    \item Точность выполнения ограничений (максимальная и средняя невязка);
    \item Качество управления (значение функции стоимости);
    \item Время вычислений на один шаг управления.
\end{itemize}

\section{Сравнение с классическим MPPI}

% TODO: провести сравнение Gauss-MPPI с классическим MPPI по всем метрикам
% TODO: представить результаты в виде таблиц и графиков

\section{Сравнение с альтернативными методами}

% TODO: сравнить с CBF-MPPI (Shield-MPPI, GS-MPPI)
% TODO: сравнить со штрафными методами
% TODO: сравнить с pi-MPPI

\section{Анализ чувствительности к параметрам}

% TODO: исследовать влияние числа траекторий $K$
% TODO: исследовать влияние температуры $\lambda$
% TODO: исследовать влияние шага интегрирования $\Delta t$

\section{Обсуждение результатов}

% TODO: обобщить результаты экспериментов
% TODO: обсудить преимущества и ограничения предложенного метода
% TODO: определить область применимости Gauss-MPPI
