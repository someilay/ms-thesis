\newpage
\begin{center}
  \textbf{\large 1. ЛИТЕРАТУРНЫЙ ОБЗОР}
\end{center}
\refstepcounter{chapter}
\addcontentsline{toc}{chapter}{1. ЛИТЕРАТУРНЫЙ ОБЗОР}

\section{Теоретические основы Path Integral Control}

Теория управления на основе интегралов по траекториям берёт начало в работах Каппена~\cite{kappen2005path}, который показал связь между стохастическим оптимальным управлением и статистической механикой. Для определённого класса задач (линейная зависимость динамики от управления, квадратичная стоимость управления) уравнение Гамильтона--Якоби--Беллмана преобразуется в линейное уравнение, решение которого выражается через интеграл по траекториям:
\begin{equation}
    u^*(x, t) = \mathbb{E}\left[ u(x, t) \cdot \exp\left(-\frac{1}{\lambda} S(\tau)\right) \right],
\end{equation}
где $S(\tau)$ --- стоимость траектории $\tau$, $\lambda$ --- параметр температуры.

Теодору и др.~\cite{theodorou2010generalized} обобщили этот подход на более широкий класс систем и предложили алгоритм PI$^2$ (Policy Improvement with Path Integrals) для обучения с подкреплением.

\section{Model Predictive Path Integral Control}

Метод MPPI был предложен Уильямсом и др.~\cite{williams2017mppi} как подход к модельно-предиктивному управлению, основанный на стохастической выборке траекторий. В отличие от градиентных методов, MPPI генерирует множество случайных траекторий и вычисляет оптимальное управление путём взвешенного усреднения:
\begin{equation}
    u^* = \sum_{k=1}^{K} w_k \cdot u_k, \quad w_k = \frac{\exp(-\frac{1}{\lambda} S_k)}{\sum_{j=1}^{K} \exp(-\frac{1}{\lambda} S_j)},
\end{equation}
где $K$ --- число траекторий, $S_k$ --- стоимость $k$-й траектории.

Ключевые преимущества MPPI:
\begin{itemize}
    \item Не требует вычисления градиентов функции стоимости;
    \item Работает с невыпуклыми и негладкими функциями стоимости;
    \item Эффективно параллелизуется на GPU~\cite{williams2017aggressive}.
\end{itemize}

MPPI успешно применяется в автономном вождении~\cite{williams2017aggressive}, управлении квадрокоптерами и манипуляторами.

\subsection{Постановка задачи и алгоритм MPPI}

MPPI решает задачу стохастического оптимального управления для систем вида:
\begin{equation}
\begin{aligned}
    & dx = f(x, u) dt + B(x) dw \\
    & x(t_0) = x_0
\end{aligned}
\end{equation}
где $x \in \mathbb{R}^n$ --- вектор состояния, $u \in \mathbb{R}^m$ --- вектор управления, $f$ --- функция динамики, $B$ --- матрица диффузии, $dw$ --- Винеровский процесс.

Цель --- минимизация функционала стоимости на горизонте $T$:
\begin{equation}
    J(u) = \phi(x_T) + \int_{t_0}^{t_0 + T} \mathcal{L}(x_t, u_t)\,dt,
\end{equation}
где $\phi(x_T)$ --- терминальная стоимость, $\mathcal{L}(x, u)$ --- мгновенная стоимость.

Параметр $\lambda > 0$ регулирует <<температуру>> распределения: при $\lambda \to 0$ алгоритм выбирает траекторию с минимальной стоимостью, при $\lambda \to \infty$ --- равномерное усреднение.

\section{Обработка ограничений в MPPI}

Исходная формулировка MPPI не предусматривает явной обработки ограничений. Существующие подходы классифицируются следующим образом.

\textbf{1. Штрафные методы.} Наиболее простой подход --- добавление штрафов за нарушение ограничений в функцию стоимости~\cite{williams2017mppi}. Недостатки: не гарантируют выполнение ограничений, требуют настройки весовых коэффициентов.

\textbf{2. Методы на основе барьерных функций.} Shield-MPPI~\cite{shieldmppi2024} использует Control Barrier Functions (CBF) для фильтрации небезопасных управлений. GS-MPPI~\cite{gsmppi2024} применяет композитные CBF для систем с множественными ограничениями. Эти методы гарантируют выполнение ограничений типа неравенств, но увеличивают вычислительную сложность.

\textbf{3. Вероятностные методы.} SC-MPPI~\cite{scmppi2023} встраивает фильтр безопасности в процесс выборки траекторий. BC-MPPI~\cite{bcmppi2024} присваивает вероятность допустимости каждой траектории и корректирует веса соответственно.

\textbf{4. Проекционные методы.} $\pi$-MPPI~\cite{pimppi2024} использует проекцию для обеспечения ограничений на величину и производные управления.

\section{Принцип наименьшего принуждения Гаусса}

Принцип наименьшего принуждения Гаусса утверждает: из всех кинематически возможных движений системы реализуется то, для которого величина принуждения минимальна. Математически принцип формулируется как задача квадратичного программирования:
\begin{equation}
\begin{aligned}
    & \min_{\dot{v}} && [\dot{v} - a]^T M [\dot{v} - a] \\
    & \text{s.t.} && A\dot{v} = b, \\
    & && a = M^{-1}Q,
\end{aligned}
\end{equation}
где $M$ --- матрица масс, $Q$ --- вектор обобщённых сил, $a = M^{-1}Q$ --- ускорение системы без ограничений (свободное ускорение).

\section{Уравнение Удвадия--Калабы}

Удвадия и Калаба~\cite{udwadia1992, udwadia2002} разработали универсальный метод получения уравнений движения для систем с ограничениями вида $A(q, v, t)\dot{v} = b(q, v, t)$. Решение задачи минимизации принуждения имеет аналитический вид:
\begin{equation}
    \dot{v} = \dot{v}_{\text{free}} + M^{-1/2}(AM^{-1/2})^{+}(b - A\dot{v}_{\text{free}}),
\end{equation}
где $(\cdot)^+$ --- псевдообратная матрица Мура--Пенроуза.

Уравнение применимо к:
\begin{itemize}
    \item Голономным ограничениям: $\phi(q, t) = 0$;
    \item Неголономным ограничениям: $\phi(q, v, t) = 0$;
    \item Ограничениям в операционном пространстве: $J(q)\dot{v} + \dot{J}v = \ddot{x}_d$.
\end{itemize}

\section{Выводы из обзора}

Анализ литературы позволяет сделать следующие выводы:

\begin{enumerate}
    \item Существующие методы обработки ограничений в MPPI ориентированы преимущественно на ограничения типа неравенств (безопасность, границы управления).
    
    \item Ограничения равенств на уровне ускорений (Пфаффовы ограничения) в контексте MPPI практически не исследованы.
    
    \item Принцип наименьшего принуждения Гаусса и уравнение Удвадия--Калабы предоставляют аналитическое решение для проекции на множество допустимых ускорений.
    
    \item Связь между MPPI и принципом Гаусса не исследована.
\end{enumerate}
